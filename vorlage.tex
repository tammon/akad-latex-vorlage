% Vorgaben Assignment aus Studienheft SQL03
% Formatvorgaben fuer den Text
% Umfang: 8 - 10 Seiten (inkl. Abbildungen und Tabellen, aber ohne Deckblatt, % Gliederung und Literaturverzeichnis, Eidesstattliche Erklaerung)
% Zeilenabstand: 1,5
% Schriftart: frei
% Schriftgrad: 12 pt
% Variablen, physikalische Groessen und Funktionszeichen werden kursiv gedruckt.
% Korrekturrand: links: 4,5 cm, rechts 2,0 cm, oben und unten jeweils 3,0 cm
% Deckblatt: (Adresse, AKAD-E-Mail-Adresse, Immatrikulationsnummer, Modul-
% bezeichnung, Thema, Datum, Felder für Korrektor)
% Gliederung (1 Seite)
% Literaturverzeichnis (3 - 5 Literaturquellen  z. B. Lehrbuecher, aktuelle Fachartikel recherchieren)
% Eidesstattliche Erklaerung (unterschrieben und fest eingebunden)
% Bearbeitungsdauer: 2 Monate


%
%% Document Class (Koma Script) -----------------------------------------
%% Doc: scrguien.pdf
\documentclass[%
   %draft=true,     % draft mode (no images, layout errors shown)
   draft=false,     % final mode 
%%% --- Paper Settings ---
   paper=a4,
   paper=portrait, % landscape
   pagesize=auto, % driver
%%% --- Base Font Size ---
   fontsize=12pt,%
%%% --- Koma Script Version ---
   version=last, %
%%% --- Global Package Options ---
   ngerman, % language (passed to babel and other packages)
   parskip,
   numbers=noenddot,
   bibliography=totoc
]{scrreprt} % Classes: scrartcl, scrreprt, scrbook\usepackage[ngerman]{babel}

\newif\ifappendix
\newif\iflistoffigures
\newif\iflistoftables
\newif\ifacronym
\newif\iflistofformeln
\newif\ifassignment
\newif\ifabschlussarbeit
\newif\ifglossary

% Einstellungen für das Gesamtdokument
%%% Themenbezogene Daten (bei jedem neuen Assignment zu aktualisieren):

%Titel
\newcommand*{\Titel}{Thema des Assignments}

%PDF Beschreibung
\newcommand*{\pdfsubject}{Eine kurze Beschreibung, worum es geht}

%Betreff
\newcommand*{\Arbeitstyp}{Assignment im Modul ABC01}

%Betreuer
\newcommand*{\Betreuer}{Prof. Dr. Mustermann}

%Bearbeitungszeit
\newcommand*{\Bearbeitungszeit}{8 Wochen}

%PDF Keywords
\newcommand*{\pdfkeywords}{akad, assignment, meta, information, pdf, hyperref, latex}

%%% personenbezogene Daten (bleiben gleich von Assignment zu Assignment):

%Vor- und Nachname
\newcommand*{\Name}{Max Mustermann}

%Straße und Hausnummer
\newcommand*{\Strasse}{Musterstr. 1a} 

%Plz und Ort
\newcommand*{\PlzOrt}{12345 Musterhausen} 

%Email 
\newcommand*{\Email}{max.mustermann@akad.de} 

%Immatrikulationsnummer
\newcommand*{\Immatrikulationsnummer}{123456}

%Studiengang
\newcommand*{\Studiengang}{IT-Management}

%Akademischer Grad
\newcommand*{\Grad}{Master of Science (M. Sc.)} 
%\newcommand*{\Grad}{Master of Engineering (M. Eng.)} 
%\newcommand*{\Grad}{Bachelor of Science (B. Sc.)} 
%\newcommand*{\Grad}{Bachelor of Engineering (B. Eng.)} 

%Überschrift des Literaturverzeichnisses
\newcommand*{\prefbiblioname}{Literaturverzeichnis}

%%% Nicht benötigte Zeilen mit % auskommentieren oder löschen:

%% Anhang 
\appendixtrue

%% Verzeichnisse 
%
%% Abbildungsverzeichnis 
\listoffigurestrue
%% Tabellenverzeichnis
\listoftablestrue
%% Abkürzungsverzeichnis
\acronymtrue
%% Glossar
\glossarytrue

%% Formelverzeichnis
%\listofformelntrue

%% Diese Vorlage wird für ein Assignment benutzt (statt für eine Abschlussarbeit)
\assignmenttrue


% Allgemeine Präambel für die Einbindung von Paketen
\input{preamble/preamble}

% Sonstige Hilfsfunktionen
%% Definition for Codeschnipsel im Fließtext
\newcommand{\code}{\texttt}

%% Todos mithilfe eines Rahmens hervorheben
\newcommand{\todo}[1]{\fbox{\parbox{\textwidth-\fboxsep*6-\unitlength}{\textbf{To do:} #1}}}

%% eine Fußnote als Platzhalter für ein Zitat
\newcommand{\citationneeded}{\footnote{\textit{missing citation}}}


% Style Einstellungen
%% Für Codeblöcke mit Syntax-Highlighting
%% http://www.ctan.org/tex-archive/macros/latex/contrib/minted/
%% Einkommentieren fuer Minted Syntax Highlighting
%\usepackage{minted}
%\definecolor{bg}{rgb}{0.95,0.95,0.95}

\makeatother

\ifassignment
\geometry{a4paper, left=25mm, right=20mm, top=30mm, bottom=30mm}
\else
\geometry{a4paper, left=45mm, right=20mm, top=30mm, bottom=30mm}
\fi

\pagenumbering{roman}


\usepackage[automark,headsepline]{scrlayer-scrpage}

\clearpairofpagestyles
\cfoot[\pagemark]{\pagemark}
\lehead{\headmark}
\rohead{\headmark}

\pagestyle{scrheadings}

\newacronym{url}{URL}{Uniform Resource Locator}
\newacronym{css}{CSS}{Cascading Style Sheets}
\newacronym{mituni}{MIT}{Massachusetts Institute of Technology}
\newacronym{abk}{Abk}{Abkürzung}
\newacronym{Abk}{Abk}{Abkürzung}

\newglossaryentry{pi}{
name=$\pi$,
description={Die Kreiszahl},
sort=Pi
}

\addbibresource{literatur.bib}

% BEGIN DOCUMENT %%%%%%%%%%%%%%%%%%%%%%%%%%%%%%%%%%%%%%%%%%%%%%%%%%%%%%%%%%%%%%%
\begin{document}
%%%%%%%%%%%%%%%%%%%%%%%%%%%%%%%%%%%%%%%%%%%%%%%%%%%%%%%%%%%%%%%%%%%%%%%%%%%%%%%%

% TITEL PAGE %%%%%%%%%%%%%%%%%%%%%%%%%%%%%%%%%%%%%%%%%%%%%%%%%%%%%%%%%%%%%%%%%%%

\begin{titlepage}
\newgeometry{left=25mm, right=20mm, top=30mm, bottom=30mm}
\begin{center}
\thispagestyle{empty}

\Huge{\Titel}
\vspace{2cm}
\onehalfspacing

\Large{\textbf{\Arbeitstyp}}

\vspace{1cm}
\normalsize

von

\vspace{.5cm} 
\large{\Name}
\normalsize
\vspace{1cm}

\ifassignment
\else
Zur Erlangung des akademischen Grades \\
\textbf{\Grad}
\vspace{1cm}
\fi

Im Studiengang \Studiengang \\
an der staatlich anerkannten AKAD Hochschule Stuttgart
\vspace{2cm}

\today

\vspace{2cm}

\includegraphics[scale=0.35]{img/akad_logo.png}

\end{center}

\vfill
\begin{spacing}{1.2}
    \begin{tabbing}
	    \hspace{9cm}     \= \kill
	    \textbf{Bearbeitungszeit}  \>  \Bearbeitungszeit \\
	    \textbf{Betreuer}              \>  \Betreuer \\
	    \textbf{Immatrikulationsnummer}  \>  \Immatrikulationsnummer \\
	    \textbf{E-Mail}		\> \href{mailto:\Email}{\Email} \\
	    \textbf{Adresse}		\> \Strasse \\
	    		\> \PlzOrt
	\end{tabbing}
\end{spacing}
\restoregeometry
\end{titlepage}

%%%%%%%%%%%%%%%%%%%%%%%%%%%%%%%%%%%%%%%%%%%%%%%%%%%%%%%%%%%%%%%%%%%%%%%%%%%%%%%%

\normalsize

\begin{spacing}{1.0} % Verzeichnisse werden mit einzeiligem Abstand gesetzt

% Inhaltsverzeichnis %%%%%%%%%%%%%%%%%%%%%%
\tableofcontents

% Abbildungsverzeichnis %%%%%%%%%%%%%%%%%%%%%%
\iflistoffigures
\listoffigures 
\fi

% Tabellenverzeichnis %%%%%%%%%%%%%%%%%%%%%%
\iflistoftables
\listoftables
\fi

% Abkürzungsverzeichnis %%%%%%%%%%%%%%%%%%%%%%
\ifacronym
\printglossary[type=\acronymtype,title=Abkürzungsverzeichnis]
\fi

\ifglossary
\printglossary[title=Glossar]
\fi

% Formelverzeichnis %%%%%%%%%%%%%%%%%%%%%%
\iflistofformeln
\listof{Formel}{Formelübersicht}
\newpage
\fi


\end{spacing} 

\clearpage

\newcounter{romanPagenumber} 
\setcounter{romanPagenumber}{\value{page}} % Roemische Seitenanzahl speichern.

\nocite{*} 

\pagenumbering{arabic}

\begin{spacing}{1.5} % Zeilenabstand: 1,5 fuer den Textteil

% Einleitung
\chapter{Einleitung}
\section{Einführung in das Thema}
Lorem ipsum dolor sit amet, consetetur sadipscing elitr, sed diam nonumy eirmod tempor invidunt ut labore et dolore magna aliquyam erat, sed diam voluptua. At vero eos et accusam et justo duo dolores et ea rebum. Stet clita kasd gubergren, no sea takimata sanctus est Lorem ipsum dolor sit amet. Lorem ipsum dolor sit amet, consetetur sadipscing elitr, sed diam nonumy eirmod tempor invidunt ut labore et dolore magna aliquyam erat, sed diam voluptua. At vero eos et accusam et justo duo dolores et ea rebum. Stet clita kasd gubergren, no sea takimata sanctus est Lorem ipsum dolor sit amet.

\section{Problemstellung und Ziel dieser Arbeit}

Lorem ipsum dolor sit amet, consetetur sadipscing elitr, sed diam nonumy eirmod tempor invidunt ut labore et dolore magna aliquyam erat, sed diam voluptua. At vero eos et accusam et justo duo dolores et ea rebum. Stet clita kasd gubergren, no sea takimata sanctus est Lorem ipsum dolor sit amet. Lorem ipsum dolor sit amet, consetetur sadipscing elitr, sed diam nonumy eirmod tempor invidunt ut labore et dolore magna aliquyam erat, sed diam voluptua. At vero eos et accusam et justo duo dolores et ea rebum. Stet clita kasd gubergren, no sea takimata sanctus est Lorem ipsum dolor sit amet.

\section{Aufbau der Arbeit}

\todo{An die fertige Arbeit anpassen} % chktex -10 this is an example. When writing your own TODOs, don't include this comment and you'll get a warning about every open TODO.

Lorem ipsum dolor sit amet, consetetur sadipscing elitr, sed diam nonumy eirmod tempor invidunt ut labore et dolore magna aliquyam erat, sed diam voluptua. At vero eos et accusam et justo duo dolores et ea rebum. Stet clita kasd gubergren, no sea takimata sanctus est Lorem ipsum dolor sit amet. Lorem ipsum dolor sit amet, consetetur sadipscing elitr, sed diam nonumy eirmod tempor invidunt ut labore et dolore magna aliquyam erat, sed diam voluptua.


%Grundlagen
\chapter{Grundlagen}
\section{Text}
Das ist eine \gls{abk}, die bei der zweiten Verwendung nur noch in der Kurzform \gls{abk} angezeigt wird.

\subsection{Fett}
\textbf{Lorem ipsum dolor sit amet, consetetur sadipscing elitr, sed diam nonumy eirmod tempor invidunt ut labore et dolore magna aliquyam erat, sed diam voluptua.}

\subsection{Kursiv}
\textit{Lorem ipsum dolor sit amet, consetetur sadipscing elitr, sed diam nonumy eirmod tempor invidunt ut labore et dolore magna aliquyam erat, sed diam voluptua.}

\subsection{Unterstrichen}
Lorem ipsum \underline{dolor} sit amet, consetetur sadipscing elitr, sed diam nonumy eirmod \underline{tempor} invidunt ut labore et dolore magna \underline{aliquyam} erat, sed diam voluptua.

\section{Fußnote}

Text mit Fußnote \footnote{Die Fußnote zum Text} 

\section{Zitate}

Dies ist ein ganz kurzer Beispieltext \footnote{\cite{Richter2016}}. Und noch ein Zitat ohne Fußnote: \cite{Jacobsen2017}
\\
Zitate auf Webseiten: \cite{PlutoRed}
\\
Einträge ins Literaturverzeichnis (im File literatur.bib) können manuell geschrieben oder durch tools erzeugt werden, z.B. \url{http://www.literatur-generator.de}

\section{Aufzählung}

\begin{itemize}
\item\textit{Punkt 1:} Text
\item Punkt 2: Text
\item Punkt 3: \\ Text
\end{itemize}

\section{Abkürzungen}
Hier werden \emph{\gls{Abk}} aus dem Glossar aufgerufen, bei der zweiten Verwendung von \emph{\gls{Abk}} wird nur die Abkürzug selbst ausgegeben.


%Hauppteil
\include{content/hauptteil}

%Schluss
\chapter{Bewertung}

\section{Zusammenfassung}
Lorem ipsum dolor sit amet, consetetur sadipscing elitr, sed diam nonumy eirmod tempor invidunt ut labore et dolore magna aliquyam erat, sed diam voluptua. At vero eos et accusam et justo duo dolores et ea rebum. Stet clita kasd gubergren, no sea takimata sanctus est Lorem ipsum dolor sit amet.

\section{kritische Würdigung}
Lorem ipsum dolor sit amet, consetetur sadipscing elitr, sed diam nonumy eirmod tempor invidunt ut labore et dolore magna aliquyam erat, sed diam voluptua. At vero eos et accusam et justo duo dolores et ea rebum. Stet clita kasd gubergren, no sea takimata sanctus est Lorem ipsum dolor sit amet.

\section{Ausblick}
Lorem ipsum dolor sit amet, consetetur sadipscing elitr, sed diam nonumy eirmod tempor invidunt ut labore et dolore magna aliquyam erat, sed diam voluptua. At vero eos et accusam et justo duo dolores et ea rebum. Stet clita kasd gubergren, no sea takimata sanctus est Lorem ipsum dolor sit amet.

\section{Erfolgsfaktoren}
Lorem ipsum dolor sit amet, consetetur sadipscing elitr, sed diam nonumy eirmod tempor invidunt ut labore et dolore magna aliquyam erat, sed diam voluptua. At vero eos et accusam et justo duo dolores et ea rebum. Stet clita kasd gubergren, no sea takimata sanctus est Lorem ipsum dolor sit amet.

\end{spacing}

\clearpage

\pagestyle{plain}


% Anhang 
\ifappendix
\appendix
\chapter{Anhang}

\begin{figure}[H]
\begin{center}
\includegraphics[scale=0.5]{resources/akad_bild1.jpg}
\caption[Akad Anhang]{Akad Anhang. Quelle: www.akad.de}
\end{center}
\end{figure}

%\section{Example Appendix}
%\label{app:example}
%\includepdf[pages=-]{resources/example.pdf}
\clearpage
\fi

% Literaturverzeichniss - Ab hier wieder Roemische Seitenzahlen

\pagenumbering{roman}
\setcounter{page}{\theromanPagenumber}
%\bibliographystyle{apalike}
%\bibliography{literatur}
\printbibliography[title=\prefbiblioname]
\onehalfspacing
\clearpage

\pagestyle{empty} 
\thispagestyle{empty}

\ifassignment
\else
\begin{center}
{\Large Eidesstattliche Erkl"arung}
\vspace*{4cm}\end{center}
\noindent
Ich versichere, dass ich das beiliegende Assignment selbstst"andig verfasst, keine anderen als die angegebenen Quellen und Hilfsmittel benutzt sowie alle w"ortlich oder sinngem"a"s "ubernommenen Stellen in der Arbeit gekennzeichnet habe. 
\vspace{3cm}

\hspace{-0.8cm}
\rule[0.5ex]{6.5cm}{1pt}
\hspace{1.3cm}
\rule[0.5ex]{6.5cm}{1pt}
(Datum, Ort)
\hspace{6.3cm}(Unterschrift)

\clearpage

%Messbox zur Druckkontrolle:
\newcommand{\Messbox}[2]{% Parameters: #1=Breite, #2=Hoehe
\setlength{\unitlength}{1.0mm}%
\begin{picture}(#1,#2)%
\linethickness{0.05mm}%
\put(0,0){\dashbox{0.2}(#1,#2)%
{\parbox{#1mm}{%
\centering\footnotesize 
%{\bf MESSBOX}\\ 
Breite $ = #1 {\ mm}$\\
H\"ohe $ = #2 {\ mm}$
}}}\end{picture}
}

\begin{center}
\textbf{--- Druckgröße kontrollieren! ---}
\\
\Messbox{100}{50} % Angabe der Breite/Hoehe in mm
\\
\textbf{--- Diese Seite nach dem Druck entfernen! ---}
\end{center}

\fi

\end{document}

