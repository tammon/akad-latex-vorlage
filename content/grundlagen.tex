\chapter{Grundlagen}
\section{Text}
Lorem ipsum dolor sit amet, consetetur sadipscing elitr, sed diam nonumy eirmod tempor invidunt ut labore et dolore magna aliquyam erat, sed diam voluptua.

\subsection{Fett}
\textbf{Lorem ipsum dolor sit amet, consetetur sadipscing elitr, sed diam nonumy eirmod tempor invidunt ut labore et dolore magna aliquyam erat, sed diam voluptua.}

\subsection{Kursiv}
\textit{Lorem ipsum dolor sit amet, consetetur sadipscing elitr, sed diam nonumy eirmod tempor invidunt ut labore et dolore magna aliquyam erat, sed diam voluptua.}

\subsection{Unterstrichen}
Lorem ipsum \underline{dolor} sit amet, consetetur sadipscing elitr, sed diam nonumy eirmod \underline{tempor} invidunt ut labore et dolore magna \underline{aliquyam} erat, sed diam voluptua.

\section{Fußnote}

Text mit Fußnote\footnote{Die Fußnote zum Text}

\section{Zitate}

Dies ist ein ganz kurzer Beispieltext\footnote{\cite{Richter2016}}. Und noch ein Zitat ohne Fußnote:~\cite{Jacobsen2017}
\\
Zitate auf Webseiten:~\cite{PlutoRed}
\\
Einträge ins Literaturverzeichnis (im File literatur.bib) können manuell geschrieben\citationneeded{} oder durch tools erzeugt werden, z.B. \url{http://www.literatur-generator.de}

\section{Aufzählung}

\begin{itemize}
  \item\textit{Punkt 1:} Text
  \item Punkt 2: \\ Text
\end{itemize}

\section{Glossar}
Hier werden \glspl{abk} aus dem Abkürzungsverzeichnis aufgerufen; bei der ersten Verwendung wird der volle Begriff ausgegeben und mit der Abkürzung in Verbindung gebracht, danach nur noch \emph{\gls{abk}} verwendet.

\gls{lorem} dolor sit amet, consetetur sadipscing elitr.
