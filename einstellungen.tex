%%% Themenbezogene Daten (bei jedem neuen Assignment zu aktualisieren):

%Titel
\newcommand*{\Titel}{Thema des Assignments}

%PDF Beschreibung
\newcommand*{\pdfsubject}{Eine kurze Beschreibung, worum es geht}

%Betreff
\newcommand*{\Arbeitstyp}{Assignment im Modul ABC01}

%Betreuer
\newcommand*{\Betreuer}{Prof. Dr. Mustermann}

%Bearbeitungszeit
\newcommand*{\Bearbeitungszeit}{8 Wochen}

%PDF Keywords
\newcommand*{\pdfkeywords}{akad, assignment, meta, information, pdf, hyperref, latex}

%%% personenbezogene Daten (bleiben gleich von Assignment zu Assignment):

%Vor- und Nachname
\newcommand*{\Name}{Max Mustermann}

%Straße und Hausnummer
\newcommand*{\Strasse}{Musterstr. 1a} 

%Plz und Ort
\newcommand*{\PlzOrt}{12345 Musterhausen} 

%Email 
\newcommand*{\Email}{max.mustermann@akad.de} 

%Immatrikulationsnummer
\newcommand*{\Immatrikulationsnummer}{123456}

%Studiengang
\newcommand*{\Studiengang}{IT-Management}

%Akademischer Grad
\newcommand*{\Grad}{Master of Science (M. Sc.)} 
%\newcommand*{\Grad}{Master of Engineering (M. Eng.)} 
%\newcommand*{\Grad}{Bachelor of Science (B. Sc.)} 
%\newcommand*{\Grad}{Bachelor of Engineering (B. Eng.)} 

%Überschrift des Literaturverzeichnisses
\newcommand*{\prefbiblioname}{Literaturverzeichnis}

%%% Nicht benötigte Zeilen mit % auskommentieren oder löschen:

%% Anhang 
\appendixtrue

%% Verzeichnisse 
%
%% Abbildungsverzeichnis 
\listoffigurestrue
%% Tabellenverzeichnis
\listoftablestrue
%% Abkürzungsverzeichnis
\acronymtrue
%% Glossar
\glossarytrue

%% Formelverzeichnis
%\listofformelntrue

%% Diese Vorlage wird für ein Assignment benutzt (statt für eine Abschlussarbeit)
\assignmenttrue
